\newpage

\section{State of the art}

Nowadays, the world has nearly as many cell phone subscriptions as inhabitants\footnote{http://www.huffingtonpost.com/2012/10/11/cell-phones-world-subscribers-six-billion\_n\_1957173.html}. For the first time, the majority of humanity is linked and has a voice. Consequently, plenty of phone communications are being generated continuously everywhere, and, what is more relevant, they are being tracked: geolocation, start/end times... This is the key, mobile phone companies record data which are very closely associated with behaviour of people.
\\
\\
Analyzing these data in a proper way discloses a great deal of social knowledge (behaviour modeling, people mobility patterns, trends and outliers) which can be applied in countless and different areas\footnote{http://www.insead.edu/v1/gitr/wef/main/fullreport/files/Chap1/1.6.pdf}: transportation, urban planning, commuting, tourism, traffic congestion, demography, sociology, economy, advertising and commerce, public health... Even without Internet connections (e-mail, IMS and so on), that is, focusing only on speech-calls and text messages, there is a vast amount of information which can be 'read' to reach further conclusions. The ability to understand the patterns of human life by analyzing the digital traces that we leave behind will transform the world, specially poor nations. Reality mining of behavior data is just beginning.
\\
\\
Let's describe a really interesting project \citep{eage09} about behavioural data. Collecting communication traces into a organization and studying the underlaying patterns, some key outcomes of interest are revealed: social network structure, inferring friendship and proximity levels, individual satisfaction... With temporal data such as call logs, location, phone status, near bluetooth devices, cell antenna ids, application usage(e-mail) and comparing these behavioural data with traditional self-report data show important conclusions.
\\
\\
Regarding D4D datasets (there are only 4 and contain really simple data), note how they have caused many and varied studies from all teams. As far as we are concerned, we discussed about several ideas: antennas network optimization in traffic terms, geospatial-temporal detection of real use for public services (hospitals, schools, police stations...), commuting patterns detection and the like.
\\
\\
Precisely, it has been the human urban mobility approach the one we chose as the core of our project. Why ?? Because it is a reality very tied to ordinary people daily lives, so that its study can reveal clues to improve people quality of life.
\\
\\
Here below a few current researches showing how identified commuting patterns are really useful to understand human motion dynamics better and to perform accurate plans and actions:
\\
\\
a) Exploring spatio-temporal commuting patterns in a Moscow university environment allows making more appropriate decisions to decrease the automobile dependence of students, promoting the non-motorized and public transportation. It is a green initiative looking for sustainability: reducing pollution and noise, avoiding congestion, improving public health and urban planning...
\\
\\
b) Classifying different urban areas based on their mobility patterns from mobile phone data. The results can be used to better understand this dynamic allowing more efficient environmental and transportation policies for the time being and for the future (since due to the regularity of the individual trajectories, it can be claimed that human mobility is highly predictive).
\\
\\
c) Time patterns and geospatial clustering based on mobile phone network data provide accurate statistics about mobility of people, population density and economic activity with detailed regional and time resolution.
\\
\\
d) Visual analytics system to study people's mobility patterns from mobile phone data. This tool allows to deeply analyze where, when and who for the calls of  people, allowing different kinds of aggregations.
\\
\\
As can be seen, communication data are everywhere ({\it we are social animals!!}) and they can be used to obtain really interesting and high-value findings. Imagine, once we know the nature and meaning of these data, it is as if we had access to a lot of complete, reliable and immediate surveys. Honestly, we strongly believe that the future lies in knowing how to process this kind of data to get unique results. MIT's Technology Review has recently identified {\bf reality mining on mobile communications} as one of '10 Emerging Technologies That Will Change the World'.
\\
\\
Several studies have proven the utility of Geographic Information Systems (GIS) for commuting analysis (flows intensities and directions), because of its efficiency dealing with data with a geographic component. This is the case of people displacement patterns on a particular portion of space. That, combined with its ability to represent data over the territory through geovisualization techniques, makes GIS one of the most used tools for this kind of studies. 
\\
The conclusion of all of those studies points to the valuable information that can be extracted from commuting, in terms of city growth, decisions about cities and companies placement or people disposition to migrate or dealing with long journeys to their work or leisure places.
\\
\\
There are many applications and use cases in the bibliography, including: the analysis of work commuting in a city, in terms of time and distance to locate spatial variations \citep{wang2000modeling}, the analysis of people beaviour patterns related to their leisure activities, the study of georeferenced commuting patterns to elaborate models that predict workplace contacts that result in disease transmission \citep{chrest2009using}, the study of dispersión trends or concentrations in specific studied areas, the study of average distances spent on displacements as well as spent times relating them with companies and cities location patterns, the study of the differences between mobility patterns of national and inmigrant employees \citep{llano2006localizacion}, the evaluation of optimal routes \citep{theriault1999modelling} the analysis of costs and transport problems in intraurban and interuban structures \citep{zhan2008gis}, etc.
