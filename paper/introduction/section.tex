\newpage
\setcounter{secnumdepth}{1}
\section{Introduction and motivation}
The Paradigma Labs Research Group core values are conducted by one motivation: "Figure out the dynamic and fuzzy intersection between Humanity and Thechnology".  Therefore a Challenge that main goal is the human improve and developt by means of the study of its mobile communications quickly captured our attention.
\\
Our final aim is detecting spatio-temporal patterns in order to obtain an useful knowledge to better manage the country resources. For example, if we could predict the traffic intensity segmented by road, week day and hour, then another secondary roads could be suggested or the budget for the most used ones could be increased. Trough mobile communitacions, a specific user can be tracked along the day, not only by means of "Call comunications" also by applications running on the user terminal like chats, RSS consumers, etc ... The dataset provides by Orange is a sampled one, and only apply "Call communications", however, we belive that with thw whole set of data i.e.: app and call comunications future models could be several kind of dynamics more accurate.
\\
\\
Our experience studing and moedling several kinds of Human Dynamics like ESF project DynCoopNet\citep{dyncoopnet2012} or Business Intelligence Tracking Tool on Twitter \citep{labselecciones}, has teach us to explore through two main point of view: Space and Temporal components. We belive that in mathematical model related with Human Dynamics must be present this two components. The Temporal component is useful to provides to the final research a tool to go forward and backward in order to get more deeply understand of the dynamic, The Geographical component provides a more high understand related with the human mobility across the time. Therefore our model and the sudy will carry out by this two imain components.
\\
\\
In this paper, we propose a Space-Temporal Model to achieve this goal. A Space Model, because user interactions with geolocated antennas are analyzed and treated, and a Temporal Model because several time windows are used to group this user dynamics. Later, the mix of this two variables (space and temporal) are consumed and visualized by a Geographic Information System. At the begining several results will be showed, however, the tools with the data will be available to the research comunity to study more deeply the studied dynamics.
\\
\\
Our aim i related to the Commuting concept and could be defined as follow: \emph{Commuting} is regular travel between one's place of residence and place of work or full-time study \citep{wiki:commuting}, but  sometimes its refers to \emph{any regular or often repeated traveling between locations when not work related}. Our first commuting approach is defined like: "Mobility patterns through inferring dynamic users movements grouped by temporal windows". 
\\
A dynamic user is defined like an user changing his antenna location within the studied temporal window (i.e.: each temporal window groups the whole user communication into one specific hour). Among these temporal windows, static users have been removed, i.e.: users that do not change their antennas locations within the temporal range. The justification to remove these users comes to focus our study on users that are moving into this temporal windows and perform micro-displacements.
