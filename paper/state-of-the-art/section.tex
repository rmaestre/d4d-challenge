\newpage

\section{State-of-the-art}

Nowadays, the world has nearly as many cell phone subscriptions as inhabitants\footnote{http://www.huffingtonpost.com/2012/10/11/cell-phones-world-subscribers-six-billion_n_1957173.html}. For the first time, the majority of humanity is linked and has a voice. Consequently, plenty of phone communications are being generated continuously everywhere, and, what is more relevant, they are being tracked: geolocation, start/end times... This is the key, mobile phone companies record data which are very closely associated with behaviour of people.
\\
\\
Analyzing these data in a proper way discloses a great deal of social knowledge (behaviour modeling, mobility patterns and trends, outliers discovery) which can be applied in countless and different areas\footnote{http://www.insead.edu/v1/gitr/wef/main/fullreport/files/Chap1/1.6.pdf}: transportation, urban planning, commuting, tourism, traffic congestion, demography, economy, advertising and commerce, public health... Even without Internet connections (e-mail, IMS and so on), that is, focusing only on speech-calls and text messages, there is a vast amount of information which can be 'read' to reach further conclusions. The ability to understand the patterns of human life by analyzing the digital traces that we leave behind will transform the world, specially poor nations. Reality mining of behavior data is just beginning.
\\
\\
Let’s enumerate a few examples of current projects related to behavioural data
\\
\\
As can be seen, communication data are everywhere ({\it we are social animals!!}) and they can be used to obtain really interesting and high-value findings. Imagine, once we know the nature and meaning of these data, it is as if we had access to a lot of complete, reliable and immediate surveys. Honestly, we strongly believe that the future lies in knowing how to process this kind of data to get unique results. MIT's Technology Review has recently identified {\bf reality mining on mobile communications} as one of '10 Emerging Technologies That Will Change the World'.
\\
\\
Note how the D4D datasets (there are only 4 and contain really simple data) have caused many and varied studies from all teams. As far as we are concerned, we discussed about several ideas: antennas network optimization in traffic terms, geospatial-temporal detection of real use for public services (hospitals, schools, police stations...), commuting patterns detection and the like.
\\
\\
Precisely, it has been the human urban mobility approach the one we chose as the core of our project. Why ?? Because it is a reality very tied to ordinary people daily lives, so that its study can reveal clues to improve people quality of life.
\\
\\
Here below a few current researches showing how identified commuting patterns are really useful to understand human motion dynamics better and to perform accurate plans and actions:
\\
\\
a) Exploring spatio-temporal commuting patterns in a university environment allows making more appropriate decisions to decrease the automobile dependence of students, promoting the non-motorized and public transportation. It is a green initiative looking for sustainability: reducing pollution and noise, avoiding congestion, improving public health and urban planning...
\\
\\
b) Classifying different urban areas based on their mobility patterns from mobile phone data. The results can be used to better understand this dynamic allowing more efficient environmental and transportation policies for the time being and for the future (since due to the regularity of the individual trajectories, it can be claimed that human mobility is highly predictive).
\\
\\
c) Time patterns and geospatial clustering based on mobile phone network data provide accurate statistics about mobility of people, population density and economic activity with detailed regional and time resolution.
\\
\\
d) Visual analytics system to study people's mobility patterns from mobile phone data.
\\
\\